\section{Характеристика организации}

ООО «СМОЛСТРОЙГАРАНТ» — строительная компания, занимающаяся реконструкцией и ремонтом строительных объектов. С 2016 года является подрядчиком ПАО «ГМК „Норильский никель“». Производит демонтаж и переоборудование тепловых коммуникаций, а также капитальный ремонт теплоэлектроцентралей  (ТЭЦ-1, г. Норильск).  

Практика проходила в отделе технической безопасности \cite{safety}, который занимается инструктажем и проверкой соблюдения техники безопасности на строительных объектах. В условиях крайнего севера, а также при работе с трехфазными масляными трансформаторами, пренебрежение техникой безопасности может привести к гибели, поэтому каждый работник в обязательном порядке проходит инструктаж и тестирование. 


\section{Формат прохождения практики}

Строительные объекты представляют собой среду, в которой взаимодействуют разнообразные технические процессы, большое количество людей, техники, а также техногенных объектов, являющихся источниками повышенной опасности. В связи с этим, стройплощадки могут быть, а зачастую являются местами повышенной опасности для здоровья и безопасности работников. На строительных объектах придерживаются строгих норм и правил, чтобы обеспечить безопасность рабочего персонала и минимизировать риски производственных травм, несчастных случаев.  Для выполнения задания, было необходимо проанализировать используемые способы проведения инструктажа и проверки знаний рабочих по технике безопасности на строительных объектах. 


\section{Анализ используемых способов прохождения инструктажа и проверки знаний рабочих по технике безопасности на строительных объектах}

На строительных объектах безопасность играет ключевую роль, и одним из основных инструментов обеспечения безопасности является инструктаж работников по технике безопасности. Это процесс, включающий в себя не только передачу информации, но и проверку знаний и понимания сотрудниками правил и процедур безопасности.  

Рассмотрим различные способы прохождения инструктажа и проверки знаний рабочих на строительных объектах:  
\begin{itemize}
\item Индивидуальный Инструктаж. Этот метод предполагает проведение инструктажа каждого работника индивидуально или в малых группах. Преимуществом данного подхода является персональный подход: инструктор может учесть особенности каждого сотрудника. К минусам относятся низкая скорость обучения работников, высокая нагрузка на инструктора, высокие затраты на обучение большого количества работников.

\item Групповой инструктаж. Такого рода инструктаж проводится для небольших групп работников. Его плюсами являются высокая эффективность при небольшом количестве обучающихся, высокая скорость обучения относительно индивидуального инструктажа при большом количестве обучающихся, меньшая нагрузка на инструкторов.  

\item Онлайн-Инструктаж. Метод проведение инструктажа с помощью современных технологий, таких как: Skype, Zoom. Главным преимуществом этого способа прохождения инструктажа является удобство инструкторов и работников. Инструктаж можно пройти, не тратя время на проезд до пункта проведения инструктажа, которое, в условиях крайнего севера, может занимать больше дня. К минусам относится необходимость в соответствующем оборудовании, а также стабильном интернет соединении.  
\end{itemize}

Для каждого из способов проведения инструктажа предусмотрена проверка (тестирование) знаний сотрудников.   


\section{Анализ действующих систем учета прохождения инструктажа по технике безопасности}

Данные о сотрудниках, времени и месте прохождения инструктажа и тестирования по технике безопасности хранятся в виде таблиц на бумажных носителях, в специализированных архивах. В рамках прохождения практики была поставлена задача “цифровизации” процесса хранения данных. Работать с вышеупомянутыми данными предстоит обычным офисным работникам, прорабам на строительных объектах, поэтому помимо хранения данных, необходимо реализовать интуитивно понятный интерфейс взаимодействия через смартфон.


\section{Разработка программного обеспечения}

\subsection{Выбор среды и языка программирования}
 
В современном мире существует огромное количество языков программирования. В целях удобства и простоты редактирования кода другим специалистом в будущем, был выбран язык Python. Python -- один из наиболее популярных и многофункциональных языков программирования в мире. 

Плюсы Python:
\begin{itemize}
\item Простота обучения и использования;
\item Многозадачность и поддержка объектно-ориентированного программирования;
\item Большое сообщество и экосистема;
\item Кросс--платформенность.
\end{itemize}

Минусы Python:
\begin{itemize}
\item Низкая производительность при обработке большого объема данных;
\item Сложность в разработке больших проектов в связи с динамической типизацией, потенциально приводящей к ошибкам во время выполнения.
\end{itemize}

На момент написания отчета, в компании работает порядка пятисот человек. Данное количество человек не является критичным, чтобы назвать проект большим. В связи с подобным обстоятельством, было принято решение закрыть глаза на минусы Python.

В качестве среды разработки выбрано программное обеспечение от компании Jet Brains -- PyCharm \cite{pycharm}. PyCharm -- это интегрированная среда разработки (IDE) для языка программирования Python. Он предоставляет мощные инструменты для написания, отладки и анализа кода Python, а также поддерживает множество фреймворков и библиотек. Например, асинхронный фреймворк для Telegram Bot API -- Aiogram \cite{aiogram}.

\subsection{Выбор средства работы с базами данных}

В качестве системы управления базами данных (СУБД) был выбран SQLite \cite{sqllite}. SQLite -- это бесплатная, серверная, встраиваемая система управления реляционными базами данных (СУБД). Она отличается от большинства СУБД тем, что не требует отдельного сервера и запускается непосредственно в приложении, что делает ее легкой и быстрой в использовании. SQLite хранит данные в одном файле базы данных, который можно легко переносить и резервировать.  

Особенности SQLite:
\begin{itemize}
\item Кросс-платформенность. SQLite поддерживается на различных операционных системах, включая Windows, macOS, Linux и многие другие. Это делает её идеальным выбором для кросс-платформенных приложений;
\item Нет необходимости в сервере. Поскольку SQLite является встраиваемой СУБД, она не требует отдельного сервера для своей работы. Это снижает накладные расходы на обслуживание и упрощает настройку;
\item Легковесность. SQLite имеет небольшой размер и низкие системные требования, что позволяет использовать её на ресурсоограниченных устройствах и в мобильных приложениях;
\item Поддержка SQL. SQLite поддерживает SQL-запросы, что делает её знакомой и удобной для разработчиков, знакомых с языком SQL;
\item Транзакции и ACID-совместимость. SQLite обеспечивает поддержку транзакций и соблюдает принципы ACID (Atomicity, Consistency, Isolation, Durability), обеспечивая надежность и целостность данных.
\end{itemize}

Области Применения SQLite:
\begin{itemize}
\item Мобильные приложения. SQLite широко используется для хранения данных в мобильных приложениях, работающих на платформах Android и iOS;
\item Встроенные системы. SQLite подходит для встроенных систем, таких как медицинские устройства, автомобильные системы информации и другие.
\end{itemize}


